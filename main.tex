\documentclass{article}
%\bibliographystyle{unsrtnat}

\usepackage[nohead, nomarginpar, margin=1in, foot=.25in]{geometry}

\bibliographystyle{plain}
\usepackage[tbtags]{amsmath}
\usepackage{MnSymbol}
\usepackage{xparse}
\usepackage{tabularx}
\usepackage{multirow}
\usepackage[export]{adjustbox}
\usepackage{graphicx}
\usepackage{hyperref}

\usepackage{datetime}

\newdateformat{UKvardate}{%
\THEDAY\ \monthname[\THEMONTH], \THEYEAR}
\UKvardate

%\usepackage{svg}

\usepackage[raggedright]{titlesec}

\usepackage{xcolor}
\usepackage[printwatermark]{xwatermark}
\usepackage{tikz}

%\newsavebox\mybox
%\savebox\mybox{\tikz[color=black!30,opacity=0.3]\node[align=center]{DRAFT \\ Do not distribute};}
%\newwatermark*[
%  allpages,
%  angle=45,
%  scale=8,
%  xpos=-30,
%  ypos=35
%]{\usebox\mybox}

\usepackage[parfill]{parskip}
%\parindent 0pt
%\parskip 12pt

\renewcommand{\baselinestretch}{1.15}

% \newcommands{\name}[number of args]{command} here
\newcommand{\eref}[1]{Eq.~\ref{#1}}
\newcommand{\fref}[1]{Fig.~\ref{#1}}
\newcommand{\sref}[1]{Section~\ref{#1}}
\newcommand{\srefs}[1]{Sections~\ref{#1}}

\newcommand{\dPois}[1]{\sim \textrm{Pois}(#1)}

% \def\name{definition} here
\def\eg*{\textit{e.g.}}
\def\ie*{\textit{i.e.}}
\def\nb*{\textit{n.b.}}
\def\etc*{\textit{etc.}}

\def\TPR{\textrm{TPR}}
\def\TNR{\textrm{TNR}}
\def\FPR{\textrm{FPR}}
\def\FNR{\textrm{FNR}}

\def\PPD{\textrm{PPD}}
\def\CPP{\textrm{CPP}}
\def\CPD{\textrm{CPD}}

\def\rhop{\rho_{{+}}}
\def\rhon{\rho_{{-}}}

\newcommand{\abs}[1]{\lvert #1\rvert}
%\renewcommand{\thesection}{S\arabic{section}}
%\renewcommand{\thetable}{S\arabic{table}}
%\renewcommand{\thefigure}{S\arabic{figure}}
%\renewcommand{\theequation}{S\arabic{equation}}

\usepackage{lineno}

\begin{document}

\def\partitle{Pre-vaccination testing could expand coverage of 2-dose COVID vaccines}

\author[1,2,*]{Carl A. B. Pearson}
\author[1]{Sam Clifford}
\author[2]{Juliet R. C. Pulliam}
\author[1]{\mbox{Rosalind} M. Eggo}
\affil[1]{Department of Infectious Disease Epidemiology \&\ Centre for Mathematical Modelling of Infectious Diseases,
London School of Hygiene \&\ Tropical Medicine

Keppel Street, London, United Kingdom WC1E 7HT

\{carl.pearson, sam.clifford, r.eggo\}@lshtm.ac.uk}
\affil[2]{DSI-NRF Centre of Excellence in Epidemiological Modelling and Analysis, Stellenbosch University

19 Jonkershoek Road, Stellenbosch, South Africa, 7600

pulliam@sun.ac.za
}
\affil[*]{corresponding}

\title{\partitle}

\maketitle

%\clearpage

%\tableofcontents
%\listoftables
%\listoffigures

%\clearpage

\linenumbers

\section{Overview}

Assuming the availability of a sufficiently affordable and accurate point-of-care antibody test, and confirmation of initial laboratory-based evidence that SARS-CoV-2 antibody positive individuals (\ie* seropositive individuals) receive as much protection from a single dose as seronegative individuals receive from two doses, a test-and-vaccinate programme could be effective at expanding vaccine coverage and reducing per-protected-individual costs.

In this supplement, we provide the mathematical details of a direct benefit analysis of a homogeneous population receiving the vaccine. This simplifies elements that would increase the benefit of testing (\eg* no indirect benefits due to transmission reduction), but also ignores aspects that could reduce those benefits (\eg* same seroprevalence in prioritized populations). We also assume that the target population is not potentially subject to saturation effects; \eg* we can ignore how to deal with doubling 60\% coverage.

\section{Expanded Protection}

We assume homogeneous distribution of potentially test-detectable prior SARS-CoV-2 infection that can be characterized by a population level seroprevalance, $\rhop = 1 - \rhon$. Similarly, we assume that the test can be characterized uniformly by two parameters, sensitivity (the true-positive rate, the complement of the false negative rate $\TPR = 1 - \FNR$), and specificity (the true-negative rate, the complement of the false positive rate $\TNR = 1 - \FPR$). For practical applications, these values will be entangled, since we know that \eg* SARS-CoV-2 antibody responses decline with time.

If we assume pessimistically that a single dose is not protective when given to seronegative individuals, then the people protected per dose ($\PPD$) is:

$$
\PPD = 0.5\left(\rhon * \TNR + \rhop * \FNR \right) + \rhop*\TPR + 0\left(\rhon * \FPR\right)
$$

which we can re-arrange in terms of complementary parameters:

\begin{equation}\label{ppd}
\begin{aligned}
2\PPD &= \left((1-\rhop) * \TNR + \rhop * (1-\TPR) \right) + 2\rhop*\TPR \\
&= \TNR-\rhop\TNR + \rhop - \TPR\rhop + 2\rhop*\TPR \\
&= \TNR - \rhop\TNR + \rhop + \TPR\rhop \\
&= \TNR + \rhop \left(1 + \TPR-\TNR\right)
\end{aligned}
\end{equation}

Recall that without testing, two doses are required for vaccine immunity, \ie* $\PPD=0.5$. This corresponds to the situation where $\TNR=1$ (because the ``test'' is simply that everyone gets two doses) and $\TPR=0$ (because no one gets a single dose). Relative to this baseline, the change in $\PPD$ due to introducing testing is:

$$
\Delta_{\PPD} = \frac{2\PPD\left(\TNR,\TPR,\rhop\right)}{2\PPD\left(1,0,\sim\right)} - 1 = \TNR + \rhop \left(1 + \TPR-\TNR\right) - 1
$$

For this analysis, we assume a single dose is non-protective in seronegative individuals: that helps ensure that the conclusions represent the {\em minimal} benefit, and thus decisions continue to beneficial even after accommodating real-world factors which are impractical to model. This particular limiting assumption implies that introducing testing can reduce the number of people effectively vaccinated. As practical matter, this is only a problem with extreme combinations of seroprevalence and test performance, which are practically irrelevant for the settings where this scheme is worth considering. The constraining equation is $\Delta_{\PPD} > 0$ or $\TNR + \rhop \left(1 + \TPR-\TNR\right) > 1$. For example, in a setting with 20\% seroprevalence, using a test with low specificity, 50\%, would be detrimental even with a perfectly sensitive test. 

\section{Cost}

Consider total cost per vaccine dose of $V$ (\ie* production, logistics, and administration) and total cost per test of $T$. On the margin, imagine purchasing one more protected person: that can be accomplished for $2V$ (\ie* two doses to get a fully vaccinated person without testing) or for some expected number of tests, $nT$, to allow re-allocation of unnecessary second doses from seropositive recipients.

If testing increases the $\PPD$ by \eg* 10\%, then using 20 tests (on average) would add another fully immunized individual. Even at the point where the test is perfect and everyone is seropositive (\ie* testing increases $\PPD$ by 100\%), two tests are still required: one to free up the second dose from an individual and one to confirm that the next recipient in line only needs one. The generalization is:

$$
n = \frac{2}{\Delta_{\PPD}}
$$

meaning that in direct-benefit terms, buying tests is preferred when $\Delta_{\PPD} > 0$ and

$$
\frac{T}{\Delta_{\PPD}} \le V \rightarrow \frac{T}{V} \le \Delta_{\PPD}
$$

This inequality indicates settings where using testing to more efficiently distribute vaccine courses is preferred to purchase additional courses, but does not indicate how much more cost effective testing is.

Under a test-and-vaccinate program, the cost of administering a dose is either $T+V$ (first dose) or $V$ (second dose). One immediate consequence is that for a given number of doses, introducing testing makes the average dose more costly as well as the total cost of the program. The question then is whether this additional cost is worthwhile, which can be understood in terms of the cost per person protected, rather than the cost per dose administered.

For every first dose administered, the scheme also administers second doses to individuals that test negative (accurately or not). The probability of that second dose is $\rhon\TNR + \rhop(1-\TPR)$, meaning that for every first dose there are that many second doses, and therefore the fraction of all doses that are first doses is:

$$
p(\textrm{dose is a first dose}) = \frac{1}{1+(1-\rhop)\TNR + \rhop(1-\TPR)} 
$$

Note that as in the previous section, this fraction corresponds to the no-test scheme when $\TPR=0$, $\TNR=1$; \ie*, half of doses are first doses under that scenario. We can find the expected cost per dose, $\CPD$, by weighting first and second dose prices according to their relative mix. Both have a $V$ component, and first doses have a $T$ component:

$$
\CPD = V + \frac{T}{1+(1-\rhop)\TNR + \rhop(1-\TPR)} 
$$

With the cost per dose and people protected per dose, we can also compute the cost per person protected,

$$
\CPP = \frac{\CPD}{\PPD} = 2\frac{V + \frac{T}{1+(1-\rhop)\TNR + \rhop(1-\TPR)}}{\TNR + \rhop \left(1 + \TPR-\TNR\right)}
$$

For the no-test situation, $T=0$, $\TPR=0$, and $\TNR=1$ and we can see that we recover $\CPP=2V$ under those assumptions. We can use this to calculate relative change again:

$$
\Delta_{\CPP} = \frac{\CPD}{\PPD}-1 =\frac{1 + \frac{T}{V}\left(1+(1-\rhop)\TNR + \rhop(1-\TPR)\right)^{-1}}{\TNR + \rhop \left(1 + \TPR-\TNR\right)} - 1
$$

\end{document}

% 